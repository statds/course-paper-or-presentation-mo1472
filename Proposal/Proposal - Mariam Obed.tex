\documentclass[12pt]{report}
\begin{document}
\title{Proposal: Mediation Analysis}
\author{Mariam Obed\\
  University of Connecticut
}
\date{October 10th, 2022}
\maketitle
\begin{document}
\maketitle
\paragraph{Introduction}
With the use of a third variable known as a mediator variable, statistical mediation analysis aims
to explain the link between an independent variable and dependent variable. By identifying any
missing factors required to explain the link between the two variables, mediation analysis enables
us to search for a more precise study of the impact the independent variable has on the dependent
variable. Prior to examining the impact the mediator variable has on the dependent variable, we
first evaluate the impact the independent variable has on the mediator variable. Without employing
a direct link, mediation analysis allows us to take into account other elements that define the
relationship between the independent and dependent variables. Suppose if earlier studies indicated
that a person's television viewing habits indicate the
development of lung cancer. But according to our theory, a person's daily cigarette consumption
predicts how many times they watch television, which in turn predicts the development of lung
cancer. This is an illustration of a research problem that underwent mediation analysis. However,
this paper will discuss the relationship between social media usage and academic performance by
performing statistical mediation analysis. In this sense, the mediator variable would be social
media usage, while the independent variable would be procrastination and the dependent variable
would be academic performance.
\paragraph{Specific Aims}
This paper aims to find out if social media usage is a potential mediator variable that links the
independent variable, procrastination, and the dependent variable, academic performance. There are
the different types of mediation models that we can use to analyze our research questions.
Mediation analysis is a topic that is not very well explained, whether it be a college class or a
work environment. In order to explain the relationship between an independent variable and a
dependent variable, statistical mediation analysis employs a third variable known as a mediator
variable.
\paragraph{Data}
Data will be aquired from various paper that are based on the relationship between social media
usage and academic performance. There will be three variables that will be used throughout, X, M,
and Y, where X is the independent variable, M is the mediator variable, and Y is the independent
variable. 
\paragraph{Research Design and Methods}
This paper will be using regression analysis in order to fit the various models that will be
presented. Other methods that will be used include Barron and Kenny's steps for mediation analysi
, Sobel's test for significance, and the Bootstrap method for significance. The data pool will
largely be taken from university students.
\paragraph{Discussion}
The expectations going into this paper is to observe how strong of a link there is between those
that use social media excessively and those that have poor academic performance. This paper will
most likely corroborate existing results and assumption based on many of the studies that are
present. If the investigation is not what is expected, it will be shocking to students and may
change the way that many students study.
\paragraph{Conclusion}
The purpose of this paper is to determine whether social media use serves as a possible mediator
between the dependent variable, academic achievement, and the independent variable,
procrastination. In order to assess our study topics, we might employ a variety of mediation
models.Whether in a college class or at work, mediation analysis is a subject that needs more
explanation. A third variable known as a mediator variable is used in statistical mediation
analysis to help explain the link between an independent variable and a dependent variable.
Going into this research, the goal is to determine the strength of the association between
excessive social media use and subpar academic achievement. This study will most likely confirm
current findings and presumptions based on a number of the available investigations. Students will
be shocked if the probe turns out something unexpected, which might cause many of them to alter
their study habits.
\end{document}